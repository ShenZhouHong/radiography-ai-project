% Word budget: ~300 words

% Who is the project for?
% What are you hoping to achieve for your users/audience?
% What are you hoping to learn or find out through your project?
% How will you demonstrate that your project fulfils its aims?

\chapter{Introduction}
Long bone fractures are a frequent effect of high-energy trauma \autocite{HStein1999}, among which tibial fractures of the lower extremities are the most common. These fractures require long-term follow-up, where after initial fixation the fracture site must be re-examined at regular intervals for callus formation\footnote{The development of cartilaginous material containing bone-forming cells.}, bridging, and union \autocite{Jones2020}. Whelan et al's Radiographic Union Score for Tibial Fractures (RUST score) is a discrete 12-point scale that serves a common metric for the assessment of union from the lateral and anteroposterior\footnote{i.e.\ front-to-back.} radiograph of a fracture \autocite{Whelan2010}. This project proposes a means to automate the assessment of fracture healing, by using transfer-learning to develop a machine learning model to classify radiographs according to their RUST Scores.

\section{Aims and Motivation}

Non-union and delayed union are significant complications in fracture healing, one which results in heightened morbidity, loss-of-function, and infection risk \autocite{Nicholson2021}. As a result, it is important for physicians to determine non-union events so that further treatment and corrective surgery may be taken. Although non-union may be determined through a variety of clinical assessments (e.g.\ palpation, weight-bearing), the RUST score is emerging as a quantitative radiography-based measure with high consistency \autocite{Panchoo2022}, biomechanical correlativity \autocite{Cook2018}, and good guidance for postoperative rehabilitation \autocite{Debuka2019}. However, in order to assess a fracture using the RUST score, a orthopaedic must examine at least two radiographs (one lateral, one anteroposterior) for callus formation --- a non-trivial process.

Recent advances in deep learning, coupled with the increasing availability of large radiographic datasets (e.g. CheXpert, LERA, MURA) \autocites{CheXpert2019}{LERA}{MURA2017} offer the possibility of automating the process of fracture classification. Certain research models such as Rajpurkar et al's DNN ConvNet are able to meet, or exceed radiologist-level performance for abnormality classification in specific anatomical domains \autocite{MURA2017}, and as of 2021 commercial developments are beginning to see regulatory clearance\footnote{Authorisation for real-world clinical use by national health agencies like the Food and Drug Administration (FDA), Health Canada, \emph{Conformité Européenne}, etc.} \autocite{Adams2021}.

However, much of the current available literature\footnote{A selection of which are analysed with commentary in \autoref{background}.} is focused on the mere detection and classification of fractures (i.e. abnormality detection). Such models either perform binary classification (e.g. \enquote{Is this a \emph{normal} radiograph?}), multi-class (e.g. \enquote{Is this a \emph{leg}, \emph{arm}, or \emph{knee} fracture?}), or localisation (e.g. \enquote{Where \emph{is} the fracture on this radiograph?}). Comparatively less work has been done on the \emph{characterisation} of radiographs, where the properties of a fracture are described \autocite{Tanzi2020}. The existence of this \enquote*{research gap} can be in part attributed to the absence of large radiographic datasets with quality labelling.

This gap in the field offers opportunity for further investigation, especially as it is not the mere presence of a fracture which informs medical decision-making, but rather its severity and properties. This study aims to build a machine-learning model which infers the RUST score of a fracture from a pair of radiographs, specifically utilising the technique of \emph{transfer-learning} in order to address the challenges of working with small datasets. By building a model that infers the RUST score of a fracture from it's associated radiograph, we hope to demonstrate the feasibility of such an approach, and open the door to further investigation, 

\section{Objectives and Evaluation}

% The goal of this project, is to develop an AI model that is able to take a pair of input radiographs, and output a projected RUST score. As the RUST score is a sum derived from the characterization of a fracture from four different tibial cortices\footnote{The cortex is the outermost layer of the bone.} (the anterior, posterior, lateral, and medial cortex) \autocite{Whelan2010}, the model must process input in pairs of two radiographs. 

% The objective of this project is to develop an AI model that is able to receive pairs of input radiographs, and output a projected RUST numeric rust score, as well as a heatmap of features that the model detects. As the RUST score is a sum derived from the characterization of a fracture from four different tibial cortices\footnote{The cortex is the outermost layer of the bone. The RUST Score relies on an examination of the anterior, posterior, lateral, and medial cortex of the tibia. \autocite{Whelan2010}}, the model must process input in pairs of two radiographs. As our dataset is limited to only a few thousand samples, we will be using transfer learning to train a broader, general-purpose model on our domain specific task.\footnote{See \autoref{methodology} for further information.}

The objective of this project is to develop an AI model using \emph{transfer-learning} that is able to predict the RUST score of a pair of anteroposterior and lateral radiographs. Because our primary constraint is the size of our dataset (around ~3,000 radiographs, see \hyperref[sec:dataset]{Dataset}), we will be utilising the technique of transfer-learning to demonstrate the feasibility of training AI models on datasets of this order of magnitude. We will be using the InceptionV3 model architecture \autocite{inceptionv3} as the base model with a custom classifier for our transfer-learning approach. 

First, we will build and evaluate model that is end-to-end trained without the use of transfer-learning (\hyperref[sec:protocol-i-method]{Protocol I}). This will serve as a baseline benchmark for our subsequent attempts.
Next, we will evaluate two different transfer learning approaches based on the InceptionV3 model architecture. The first model will use InceptionV3 trained with the general-purpose ImageNet dataset \autocite{imagenet} (\hyperref[sec:protocol-ii-method]{Protocol II}). The second model will use the same InceptionV3 architecture, but trained with the domain-specific RadImageNet dataset \autocite{radimagenet} (\hyperref[sec:protocol-iii-method]{Protocol III}). The development and comparative evaluation of these two base models for transfer learning will allow us to compare and contrast the use of a model pre-trained on a general-purpose dataset (ImageNet) versus a model pre-trained on a slightly smaller, but domain-specific dataset (RadImageNet). 

Afterwards, we will select the best-performing base model and find the best-performing hyperparameters. This will be done via a hyperparameter search in two steps, called \hyperref[sec:regime-i]{Regime I} and \hyperref[sec:regime-ii]{Regime II}. Regime I will be a random-search on batch size and dropout, while Regime II will be a grid search on learning rate and epsilon. Finally, the model trained with the best-performing hyperparameters will be evaluated on the hold-out test set, and the overall performance will be examined.

% \clearpage
% \subsection{Project Specification}

% Thus, the aims of this project can be summarised as the following three objectives:

% \begin{itemize}
%     \item Evaluate the performance of InceptionV3 trained with ImageNet and RadImageNet on a transfer learning task.
%     \item Develop and optimise the best-performing transfer learning model for use in the automated assessment of fracture healing through RUST scores.
%     \item Assess model performance through it's AUROC (Area Under Receiver Operating Characteristic) value.\footnote{See \ref{AUROC} for further information.}
% \end{itemize}

% https://en.wikipedia.org/wiki/Receiver_operating_characteristic