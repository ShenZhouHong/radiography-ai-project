\chapter{Implementation and Analysis}\label{implementation}

In this chapter, we will present the implementation of the study methodology. Recall that the methodology has three components. We will begin with the establishment of an initial baseline, by creating and training a classical \enquote*{shallow} convolutional neural network based upon LeCun et al.'s 1998 LeNet model. \autocite{lenet1998} This classical CNN baseline will serve as the minimal performance standard that our model will aim to surpass. Next, utilising the InceptionV3 architecture which will serve as our transfer-learning base model, we will train an end-to-end (i.e.\ without transfer learning) model on our radiography dataset. This will serve as an additional baseline that will allow us to validate the transfer-learning \emph{technique} against regular end-to-end training.

Following the establishment of these two baselines, we will proceed to begin an initial evaluation of two different transfer-learning base models. We will compare the performance of InceptionV3 trained with ImageNet weights \autocite{imagenet}, against InceptionV3 trained with RadImageNet \autocite{radimagenet} weights. This initial evaluation will help us explore whether a base model trained on the smaller, but domain-specific RadImageNet dataset will have any advantages over the larger, but general ImageNet dataset. We will select the better performing base model out of the two options, and proceed to optimize the model's hyperparameters.

Our model's hyperparameter search procedure consists of two steps, which we term hyperparameter search Regime I and hyperparameter search Regime II. As per our methodology, in Regime I we find the optimal batch size and dropout rate for our model. This is done using a stochastic search process where the hyperparameter space of the model is randomly sampled for \(t\) trials, where each trial consists of a k-fold cross-validation of the model with the selected hyperparameters. Once the optimal combination of batch size and dropout rate are found, we will set these hyperparameters as constant and proceed to the second hyperparameter search regime. In Regime II we find the optimal learning rate and epsilon value \(\epsilon\) for the Adam optimizer, by conducting a grid search over a selection of possible values.

\section{K-Fold Evaluation}

Before we begin, we must first implement our k-fold cross-validation routine. Since model performance is sensitive to the network's random weight initialisation\footnote{This is particularly true on small datasets with unbalanced classes like ours.} \autocite{Narkhede2022}, our methodology requires k-fold cross-validation to be conducted on every experiment (i.e.\ model run). My implementation of the k-fold cross-validation process consists of two parts: a function which will divide the dataset into \(k\) folds, as well as a function that runs the k-fold cross-validation on the given model.

\begin{listing}[H]
        \begin{minted}[
            baselinestretch=1.0,
            frame=lines,
            mathescape,
            autogobble,
            fontsize=\footnotesize,
            style=default,
            breaklines,
            breakbytoken
        ]{python}
        def k_fold_dataset(ds: tf.data.Dataset, k: int = 10) -> list[tuple[tf.data.Dataset, tf.data.Dataset]]:
            # First shard the given dataset into k individual folds.
            list_of_folds: list[tf.data.Dataset] = []
            for i in range(k):
                fold: tf.data.Dataset = ds.shard(num_shards=k, index=i)
                list_of_folds.append(fold)
        
            # Next, generate a list of train and validation dataset tuples
            list_of_ds_pairs: list[tuple[tf.data.Dataset, tf.data.Dataset]] = []
            for i, holdout_fold in enumerate(list_of_folds):
                ds_valid: tf.data.Dataset = holdout_fold
        
                # Select every fold except holdout_fold as the training folds
                training_folds: list[tf.data.Dataset] = list_of_folds[:i] + list_of_folds[i+1:]

                # ds_train size is $\frac{k-1}{k}$ of the original dataset
                ds_train: tf.data.Dataset = training_folds[0]
                for fold in training_folds[1:]:
                    ds_train = ds_train.concatenate(fold)
        
                ds_pair: tuple[tf.data.Dataset, tf.data.Dataset] = (ds_train, ds_valid)
                list_of_ds_pairs.append(ds_pair)
            
            return list_of_ds_pairs
        \end{minted}
    \caption{Sharding dataset for K-Fold Cross Validation (\href{https://github.com/ShenZhouHong/radiography-ai-project/blob/cf8c9e9a1f07849787a98b2fc51df690354bf194/python/common/kfold.py}{Github})}\label{listing:sharding}
\end{listing}

\begin{listing}[H]
        \begin{minted}[
            baselinestretch=1.0,
            frame=lines,
            mathescape,
            autogobble,
            fontsize=\footnotesize,
            style=default,
            breaklines,
            breakbytoken
        ]{python}
        def cross_validate(ModelClass: tf.keras.Model, ds: tf.data.Dataset, epochs: int = 50, batch_size: int = 128, k: int = 10) -> list[tf.keras.callbacks.History]:

            history_list: list[tf.keras.callbacks.History] = []
            train_valid_pairs: list[tf.data.Dataset] = k_fold_dataset(ds, k)
        
            for i, (ds_train, ds_valid) in enumerate(train_valid_pairs):
        
                tf.keras.backend.clear_session()
                model = ModelClass()
                model.compile(
                    optimizer=tf.keras.optimizers.Adam(),
                    loss=tf.keras.losses.BinaryCrossentropy(),
                    metrics=metrics
                )
                history = model.fit(
                    ds_train,
                    validation_data=ds_valid,
                    epochs=epochs,
                    batch_size=batch_size,
                )
                history_list.append(history.history)

            return history_list
        \end{minted}
    \caption{K-Fold Cross Validation Implementation}\label{listing:cross-validate}
\end{listing}

\begin{listing}[H]
    \begin{minted}[
        baselinestretch=1.0,
        frame=lines,
        mathescape,
        autogobble,
        fontsize=\footnotesize,
        style=default,
        breaklines,
        breakbytoken
    ]{python}
    def calculate_mean_metrics(kfold_metrics: list[dict[str, float]]) -> dict[str, list[float]]:
        # Initialise aggregate metrics with appropriate keys
        aggregate_metrics: dict[str, list[float]] = {}
        for fold in kfold_metrics:
            for metric in fold.keys():
                if metric not in aggregate_metrics:
                    aggregate_metrics[metric] = []

        # Calculate the average metric per epoch for every fold
        number_of_folds: int = len(kfold_metrics)
        for metric in aggregate_metrics.keys():
            number_of_epochs: int = len(kfold_metrics[0][metric])
            for epoch in range(number_of_epochs):
                # A list of every value for that given metric in this epoch across folds
                values_per_epoch: list[float] = [x[metric][epoch] for x in kfold_metrics]
                mean_per_epoch  : float = sum(values_per_epoch) / number_of_folds
                aggregate_metrics[metric].append(mean_per_epoch)

        return aggregate_metrics
    \end{minted}
\caption{Calculating Mean Metrics from K-Fold Data (\href{https://github.com/ShenZhouHong/radiography-ai-project/blob/52b2674f328c7595a32b7e4bcd2c6d4d4824e4ca/python/common/utilities.py}{Github})}\label{listing:calc-mean-metrics}
\end{listing}

\section{Establishing a Baseline}

\subsection{Shallow Convolutional Neural Network}

\begin{listing}[H]
    \begin{minted}[
        baselinestretch=1.0,
        frame=lines,
        mathescape,
        autogobble,
        fontsize=\footnotesize,
        style=default,
        breaklines,
        breakbytoken
    ]{python}
    class LeNet1998(tf.keras.Model):
        def __init__(self, **kwargs):
            super().__init__(**kwargs)

            self.input_layer: tf.Tensor = layers.InputLayer(input_shape=(299, 299, 3))
            self.data_augmentation: tf.keras.Sequential = tf.keras.Sequential([
                layers.RandomFlip(seed=RNG_SEED),
            ])

            self.lenet1999: tf.keras.Model = tf.keras.Sequential([
                layers.Conv2D(6, kernel_size=5, strides=1,  activation='tanh', padding='same'),
                layers.AveragePooling2D(),
                layers.Conv2D(16, kernel_size=5, strides=1, activation='tanh', padding='valid'),
                layers.AveragePooling2D(),
            ])

            self.classifier: tf.keras.Sequential = tf.keras.Sequential([
                layers.Flatten(),
                layers.Dense(1024, activation='relu'),
                layers.Dense(18, activation='sigmoid')
            ])

            self.model: tf.keras.Sequential = tf.keras.Sequential([
                    self.input_layer,
                    self.data_augmentation,
                    self.lenet1999,
                    self.classifier
            ])

        def call(self, inputs):
            return self.model(inputs)
    \end{minted}
\caption{The LeNet 1998 Shallow CNN Model (\href{https://github.com/ShenZhouHong/radiography-ai-project/blob/cf8c9e9a1f07849787a98b2fc51df690354bf194/python/initial-evaluation/lenet1998.ipynb}{Github})}\label{listing:lenet1998}
\end{listing}

\begin{listing}[H]
    \begin{minted}[
        baselinestretch=1.0,
        frame=lines,
        mathescape,
        autogobble,
        fontsize=\footnotesize,
        style=default,
        breaklines,
        breakbytoken
    ]{python}
    class LeNet1998(tf.keras.Model):
        def __init__(self, **kwargs):
            super().__init__(**kwargs)

            self.input_layer: tf.Tensor = layers.InputLayer(input_shape=(299, 299, 3))
            self.data_augmentation: tf.keras.Sequential = tf.keras.Sequential([
                layers.RandomFlip(seed=RNG_SEED),
            ])

            self.lenet1999: tf.keras.Model = tf.keras.Sequential([
                layers.Conv2D(6, kernel_size=5, strides=1,  activation='tanh', padding='same'),
                layers.AveragePooling2D(),
                layers.Conv2D(16, kernel_size=5, strides=1, activation='tanh', padding='valid'),
                layers.AveragePooling2D(),
            ])

            self.classifier: tf.keras.Sequential = tf.keras.Sequential([
                layers.Flatten(),
                layers.Dense(1024, activation='relu'),
                layers.Dense(18, activation='sigmoid')
            ])

            self.model: tf.keras.Sequential = tf.keras.Sequential([
                    self.input_layer,
                    self.data_augmentation,
                    self.lenet1999,
                    self.classifier
            ])

        def call(self, inputs):
            return self.model(inputs)
    \end{minted}
\caption{The LeNet 1998 Shallow CNN Model (\href{https://github.com/ShenZhouHong/radiography-ai-project/blob/cf8c9e9a1f07849787a98b2fc51df690354bf194/python/initial-evaluation/lenet1998.ipynb}{Github})}\label{listing:lenet1998}
\end{listing}

\subsection{End-to-End Training with InceptionV3}

\begin{listing}[H]
    \begin{minted}[
        baselinestretch=1.0,
        frame=lines,
        mathescape,
        autogobble,
        fontsize=\footnotesize,
        style=default,
        breaklines,
        breakbytoken
    ]{python}
    class TransferLearningModel(tf.keras.Model):
        def __init__(self, dropout_rate: float, **kwargs):
            super().__init__(**kwargs)

            self.input_layer: tf.Tensor = layers.InputLayer(input_shape=(299, 299, 3))
            self.data_augmentation: tf.keras.Sequential = tf.keras.Sequential([
                layers.RandomFlip(seed=RNG_SEED),
            ])

            self.inceptionv3: tf.keras.Model = tf.keras.applications.InceptionV3(
                include_top=False,
                weights='imagenet'
            )
            self.inceptionv3.trainable = False

            self.classifier: tf.keras.Sequential = tf.keras.Sequential([
                layers.GlobalMaxPooling2D(),
                layers.Dense(1024, activation='relu'),
                layers.Dropout(dropout_rate),
                layers.Dense( 512, activation='relu'),
                layers.Dropout(dropout_rate),
                layers.Dense( 256, activation='relu'),
                layers.Dropout(dropout_rate),
                layers.Dense(  18, activation='sigmoid')
            ])

            self.model: tf.keras.Sequential = tf.keras.Sequential([
                self.input_layer,
                self.data_augmentation,
                self.inceptionv3,
                self.classifier
            ])

        def call(self, inputs):
            return self.model(inputs)
    \end{minted}
\caption{Model Class for InceptionV3 (\href{https://github.com/ShenZhouHong/radiography-ai-project/blob/cf8c9e9a1f07849787a98b2fc51df690354bf194/python/common/model.py}{Github})}\label{listing:model-def}
\end{listing}

% Template for a TiKZ/PGFPlot Graph
\begin{figure}[H]
    \begin{tikzpicture}[trim axis left]
        % All the graphing elements are inside axis environment
        \begin{axis}[
            width=\textwidth,
            height=7cm,
            scale only axis,
            title={InceptionV3 End-to-End Trained Initial Evaluation ($k = 10 $)},
            xlabel={Epochs},
            ylabel={AUROC},
            xmin=1, xmax=50,
            ymin=0.5, ymax=1,
            grid=both,
            minor tick num=1,
            grid style=dotted,
            legend pos=north east,
            x tick label style={
              /pgf/number format/fixed,
              /pgf/number format/fixed zerofill,
              /pgf/number format/precision=0
            },
            y tick label style={
              /pgf/number format/fixed,
              /pgf/number format/fixed zerofill,
              /pgf/number format/precision=2
            },
        ]
            % First graph the validation AUROCs
            \addplot[
                color=blue,
                no markers,
                ultra thick
            ]
            table[
                col sep=comma,
                header=true,
                x=epochs,
                y=avg
            ]{data/initial-evaluations/inceptionv3_end2end_valid_auc.csv};
            \addlegendentry{Avg. Validation AUROC}

            \addplot[
                color=red,
                no markers,
                ultra thick
            ]
            table[
                col sep=comma,
                header=true,
                x=epochs,
                y=avg
            ]{data/initial-evaluations/inceptionv3_end2end_train_auc.csv};
            \addlegendentry{Avg. Training AUROC}

            % Highest average validation AUROC
            \addplot[
              color=blue,
              no marks,
              dotted,
              ultra thick,
              domain=0:50
            ]
            {
              0.692
            };
            \addlegendentry{$y = 0.692$}

            \foreach \n in {1,...,10} {
                \addplot[
                    color=blue,
                    no markers,
                    opacity=0.2,
                    thick
                ]
                table[
                    col sep=comma,
                    header=true,
                    x=epochs,
                    y=fold\n
                ]{data/initial-evaluations/inceptionv3_end2end_valid_auc.csv};
            }

            \foreach \n in {1,...,10} {
                \addplot[
                    color=red,
                    no markers,
                    opacity=0.2,
                    thick
                ]
                table[
                    col sep=comma,
                    header=true,
                    x=epochs,
                    y=fold\n
                ]{data/initial-evaluations/inceptionv3_end2end_train_auc.csv};
            }

        \end{axis}
    \end{tikzpicture}
    \caption{InceptionV3 Model Trained on Study Data.}
    \label{graph:inceptionv3-end2end}
  \end{figure}
  

\subsection{Baseline Metrics}

\section{InceptionV3 with Transfer Learning}

\subsection{Base Model Trained on RadImageNet Dataset}

% Template for a TiKZ/PGFPlot Graph
\begin{figure}[H]
    \begin{tikzpicture}[trim axis left]
        % All the graphing elements are inside axis environment
        \begin{axis}[
            width=\textwidth,
            height=7cm,
            scale only axis,
            title={InceptionV3 with RadImageNet Weights Initial Evaluation  ($k = 10 $)},
            xlabel={Epochs},
            ylabel={AUROC},
            xmin=1, xmax=50,
            ymin=0.5, ymax=1,
            grid=both,
            minor tick num=1,
            grid style=dotted,
            legend pos=north east,
            x tick label style={
              /pgf/number format/fixed,
              /pgf/number format/fixed zerofill,
              /pgf/number format/precision=0
            },
            y tick label style={
              /pgf/number format/fixed,
              /pgf/number format/fixed zerofill,
              /pgf/number format/precision=2
            },
        ]
            % First graph the validation AUROCs
            \addplot[
                color=blue,
                no markers,
                ultra thick
            ]
            table[
                col sep=comma,
                header=true,
                x=epochs,
                y=avg
            ]{data/initial-evaluations/inceptionv3_radimagenet_valid_auc.csv};
            \addlegendentry{Avg. Validation AUROC}

            \addplot[
                color=red,
                no markers,
                ultra thick
            ]
            table[
                col sep=comma,
                header=true,
                x=epochs,
                y=avg
            ]{data/initial-evaluations/inceptionv3_radimagenet_train_auc.csv};
            \addlegendentry{Avg. Training AUROC}

            % Highest average validation AUROC
            \addplot[
              color=blue,
              no marks,
              dotted,
              ultra thick,
              domain=0:50
            ]
            {
              0.706
            };
            \addlegendentry{$y = 0.706$}

            \foreach \n in {1,...,10} {
                \addplot[
                    color=blue,
                    no markers,
                    opacity=0.2,
                    thick
                ]
                table[
                    col sep=comma,
                    header=true,
                    x=epochs,
                    y=fold\n
                ]{data/initial-evaluations/inceptionv3_radimagenet_valid_auc.csv};
            }

            \foreach \n in {1,...,10} {
                \addplot[
                    color=red,
                    no markers,
                    opacity=0.2,
                    thick
                ]
                table[
                    col sep=comma,
                    header=true,
                    x=epochs,
                    y=fold\n
                ]{data/initial-evaluations/inceptionv3_radimagenet_train_auc.csv};
            }

        \end{axis}
    \end{tikzpicture}
    \caption{InceptionV3 with RadImageNet Weights}
    \label{graph:inceptionv3-radimagenet}
  \end{figure}
  

\subsection{Base Model Trained on InceptionV3 Dataset}

% Template for a TiKZ/PGFPlot Graph
\begin{figure}[H]
    \begin{tikzpicture}[trim axis left]
        % All the graphing elements are inside axis environment
        \begin{axis}[
            width=\textwidth,
            height=7cm,
            scale only axis,
            title={InceptionV3 with RadImageNet Weights Initial Evaluation  ($k = 10 $)},
            xlabel={Epochs},
            ylabel={AUROC},
            xmin=1, xmax=50,
            ymin=0.5, ymax=1,
            grid=both,
            minor tick num=1,
            grid style=dotted,
            legend pos=north east,
            x tick label style={
              /pgf/number format/fixed,
              /pgf/number format/fixed zerofill,
              /pgf/number format/precision=0
            },
            y tick label style={
              /pgf/number format/fixed,
              /pgf/number format/fixed zerofill,
              /pgf/number format/precision=2
            },
        ]
            % First graph the validation AUROCs
            \addplot[
                color=blue,
                no markers,
                ultra thick
            ]
            table[
                col sep=comma,
                header=true,
                x=epochs,
                y=avg
            ]{data/initial-evaluations/inceptionv3_radimagenet_valid_auc.csv};
            \addlegendentry{Avg. Validation AUROC}

            \addplot[
                color=red,
                no markers,
                ultra thick
            ]
            table[
                col sep=comma,
                header=true,
                x=epochs,
                y=avg
            ]{data/initial-evaluations/inceptionv3_radimagenet_train_auc.csv};
            \addlegendentry{Avg. Training AUROC}

            % Highest average validation AUROC
            \addplot[
              color=blue,
              no marks,
              dotted,
              ultra thick,
              domain=0:50
            ]
            {
              0.706
            };
            \addlegendentry{$y = 0.706$}

            \foreach \n in {1,...,10} {
                \addplot[
                    color=blue,
                    no markers,
                    opacity=0.2,
                    thick
                ]
                table[
                    col sep=comma,
                    header=true,
                    x=epochs,
                    y=fold\n
                ]{data/initial-evaluations/inceptionv3_radimagenet_valid_auc.csv};
            }

            \foreach \n in {1,...,10} {
                \addplot[
                    color=red,
                    no markers,
                    opacity=0.2,
                    thick
                ]
                table[
                    col sep=comma,
                    header=true,
                    x=epochs,
                    y=fold\n
                ]{data/initial-evaluations/inceptionv3_radimagenet_train_auc.csv};
            }

        \end{axis}
    \end{tikzpicture}
    \caption{InceptionV3 with RadImageNet Weights}
    \label{graph:inceptionv3-radimagenet}
  \end{figure}
  

\subsection{Comparison between RadImageNet and ImageNet}

\section{Hyperparameter Search}

\subsection{Hyperparameter Search Regime I}

\begin{listing}[H]
    \begin{minted}[
        baselinestretch=1.0,
        frame=lines,
        mathescape,
        autogobble,
        fontsize=\footnotesize,
        style=default,
        breaklines,
        breakbytoken
    ]{python}
    def hyperparameter_search(trials: int, kfolds: int = 6, epochs: int = 20) -> list[dict[str, Union[int, float, list[tf.keras.callbacks.History]]]]:
        search_results: list[dict[str, any]] = []

        for trial in range(trials):
            # Randomly pick hyperparameter options
            rng = np.random.default_rng()
            batch_size  : int   = rng.integers(16, 2048, endpoint=True)
            dropout_rate: float = rng.uniform(0.0, 0.5)

            # Conduct K-Fold cross-validation with given hyperparameters
            results: list[tf.keras.callbacks.History] = cross_validate(
                TransferLearningModel,
                ds_train_and_valid,
                k=kfolds
                epochs=epochs,
                batch_size=batch_size,
                model_kwargs={"dropout_rate": dropout_rate},
            )

            search_results.append({
                "batch_size"  : batch_size,
                "dropout_rate": dropout_rate,
                "history_list": k_fold_results
            })

        return search_results
    \end{minted}
\caption{Hyperparameter Search Regime I (\href{https://github.com/ShenZhouHong/radiography-ai-project/blob/master/python/hyperparam-search/regime-1.ipynb}{Github})}\label{listing:regime-1}
\end{listing}

\begin{figure}[H]
    \begin{tikzpicture}[trim axis left]
        % All the graphing elements are inside axis environment
        \begin{axis}[
            width=\textwidth,
            height=\textwidth,
            scale only axis,
            title={Regime I Hyperparameter Search Results (Trials = $108$, $k = 6$)},
            xlabel={Batch Size},
            ylabel={Dropout Rate},
            xmin=16, xmax=2048,
            ymin=0, ymax=0.5,
            grid=both,
            minor tick num=1,
            grid style=dotted,
            legend pos=north east,
            colormap name=viridis,
            point meta min=0.74,
            point meta max=0.82,
            colorbar,
            colormap/viridis,
            colorbar horizontal=true,
            colorbar style={
                xlabel={Max Valid. AUC},
                x tick label style={
                    /pgf/number format/fixed,
                    /pgf/number format/fixed zerofill,
                    /pgf/number format/precision=2,
                },
            },
            x tick label style={
              /pgf/number format/fixed,
              /pgf/number format/fixed zerofill,
              /pgf/number format/precision=0
            },
            y tick label style={
              /pgf/number format/fixed,
              /pgf/number format/fixed zerofill,
              /pgf/number format/precision=2
            },
        ]
            % CSV Data Table Plot Example
            \addplot[
                scatter,
                only marks,
                scatter src=explicit,
                scatter/use mapped color={
                    draw=mapped color,fill=mapped color
                },
                mark=*,
            ]
            table[
              col sep=comma,
              header=true,
              x=batch_size,
              y=dropout_rate,
              meta=max_val_auc
            ]{data/hypersearch/regime-1.csv};
            \addlegendentry{Hyperparameter Trial}

        \end{axis}
    \end{tikzpicture}
    \caption{Results for the Hyperparameter Search Regime I}
    \label{graph:regime_I}
\end{figure}

% Template for a TiKZ/PGFPlot Grouped Graph
% See https://tex.stackexchange.com/questions/440466/pgfplots-trim-axis-of-groupplots
% Regarding trimming
% See https://tikz.dev/pgfplots/libs-groupplots for general usage
% For some reason \foreach does not expand in filenames, pardon the long file.

\begin{figure}[H]
    \begin{tikzpicture}[trim axis group left]
        \begin{groupplot}[
            group style = {
                group size=3 by 3,
                horizontal sep = 15pt,
                vertical sep = 15pt,
                xlabels at=edge bottom,
                ylabels at=edge left,
                xticklabels at=edge bottom,
                yticklabels at=edge left,
            },
            width=0.333\textwidth-10pt,
            height=0.333\textwidth-10pt,
            scale only axis,
            xlabel={Epochs},
            ylabel={AUROC},
            xmin=1, xmax=20,
            ymin=0.5, ymax=1,
            grid=both,
            minor tick num=1,
            grid style=dotted,
            x tick label style={
              /pgf/number format/fixed,
              /pgf/number format/fixed zerofill,
              /pgf/number format/precision=0
            },
            y tick label style={
              /pgf/number format/fixed,
              /pgf/number format/fixed zerofill,
              /pgf/number format/precision=2
            },
        ]
            % New Plot Group Begins Here
            \nextgroupplot
            \addplot[
                color=blue,
                no markers,
                ultra thick
            ]
            table[
                col sep=comma,
                header=true,
                x=epochs,
                y=avg
            ]{data/hypersearch/regime-1-examples/1_valid_auc.csv};

            \addplot[
                color=red,
                no markers,
                ultra thick
            ]
            table[
                col sep=comma,
                header=true,
                x=epochs,
                y=avg
            ]{data/hypersearch/regime-1-examples/1_train_auc.csv};

            \foreach \n in {1,...,6} {
                \addplot[
                    color=blue,
                    no markers,
                    opacity=0.2,
                    thick
                ]
                table[
                    col sep=comma,
                    header=true,
                    x=epochs,
                    y=fold\n
                ]{data/hypersearch/regime-1-examples/1_valid_auc.csv};
            }
            \foreach \n in {1,...,6} {
                \addplot[
                    color=red,
                    no markers,
                    opacity=0.2,
                    thick
                ]
                table[
                    col sep=comma,
                    header=true,
                    x=epochs,
                    y=fold\n
                ]{data/hypersearch/regime-1-examples/1_train_auc.csv};
            }
            % New Plot Group Begins Here
            \nextgroupplot
            \addplot[
                color=blue,
                no markers,
                ultra thick
            ]
            table[
                col sep=comma,
                header=true,
                x=epochs,
                y=avg
            ]{data/hypersearch/regime-1-examples/2_valid_auc.csv};

            \addplot[
                color=red,
                no markers,
                ultra thick
            ]
            table[
                col sep=comma,
                header=true,
                x=epochs,
                y=avg
            ]{data/hypersearch/regime-1-examples/2_train_auc.csv};

            \foreach \n in {1,...,6} {
                \addplot[
                    color=blue,
                    no markers,
                    opacity=0.2,
                    thick
                ]
                table[
                    col sep=comma,
                    header=true,
                    x=epochs,
                    y=fold\n
                ]{data/hypersearch/regime-1-examples/2_valid_auc.csv};
            }
            \foreach \n in {1,...,6} {
                \addplot[
                    color=red,
                    no markers,
                    opacity=0.2,
                    thick
                ]
                table[
                    col sep=comma,
                    header=true,
                    x=epochs,
                    y=fold\n
                ]{data/hypersearch/regime-1-examples/2_train_auc.csv};
            }
            % New Plot Group Begins Here
            \nextgroupplot
            \addplot[
                color=blue,
                no markers,
                ultra thick
            ]
            table[
                col sep=comma,
                header=true,
                x=epochs,
                y=avg
            ]{data/hypersearch/regime-1-examples/3_valid_auc.csv};

            \addplot[
                color=red,
                no markers,
                ultra thick
            ]
            table[
                col sep=comma,
                header=true,
                x=epochs,
                y=avg
            ]{data/hypersearch/regime-1-examples/3_train_auc.csv};

            \foreach \n in {1,...,6} {
                \addplot[
                    color=blue,
                    no markers,
                    opacity=0.2,
                    thick
                ]
                table[
                    col sep=comma,
                    header=true,
                    x=epochs,
                    y=fold\n
                ]{data/hypersearch/regime-1-examples/3_valid_auc.csv};
            }
            \foreach \n in {1,...,6} {
                \addplot[
                    color=red,
                    no markers,
                    opacity=0.2,
                    thick
                ]
                table[
                    col sep=comma,
                    header=true,
                    x=epochs,
                    y=fold\n
                ]{data/hypersearch/regime-1-examples/3_train_auc.csv};
            }
            % New Plot Group Begins Here
            \nextgroupplot
            \addplot[
                color=blue,
                no markers,
                ultra thick
            ]
            table[
                col sep=comma,
                header=true,
                x=epochs,
                y=avg
            ]{data/hypersearch/regime-1-examples/4_valid_auc.csv};

            \addplot[
                color=red,
                no markers,
                ultra thick
            ]
            table[
                col sep=comma,
                header=true,
                x=epochs,
                y=avg
            ]{data/hypersearch/regime-1-examples/4_train_auc.csv};

            \foreach \n in {1,...,6} {
                \addplot[
                    color=blue,
                    no markers,
                    opacity=0.2,
                    thick
                ]
                table[
                    col sep=comma,
                    header=true,
                    x=epochs,
                    y=fold\n
                ]{data/hypersearch/regime-1-examples/4_valid_auc.csv};
            }
            \foreach \n in {1,...,6} {
                \addplot[
                    color=red,
                    no markers,
                    opacity=0.2,
                    thick
                ]
                table[
                    col sep=comma,
                    header=true,
                    x=epochs,
                    y=fold\n
                ]{data/hypersearch/regime-1-examples/4_train_auc.csv};
            }
            % New Plot Group Begins Here
            \nextgroupplot
            \addplot[
                color=blue,
                no markers,
                ultra thick
            ]
            table[
                col sep=comma,
                header=true,
                x=epochs,
                y=avg
            ]{data/hypersearch/regime-1-examples/5_valid_auc.csv};

            \addplot[
                color=red,
                no markers,
                ultra thick
            ]
            table[
                col sep=comma,
                header=true,
                x=epochs,
                y=avg
            ]{data/hypersearch/regime-1-examples/5_train_auc.csv};

            \foreach \n in {1,...,6} {
                \addplot[
                    color=blue,
                    no markers,
                    opacity=0.2,
                    thick
                ]
                table[
                    col sep=comma,
                    header=true,
                    x=epochs,
                    y=fold\n
                ]{data/hypersearch/regime-1-examples/5_valid_auc.csv};
            }
            \foreach \n in {1,...,6} {
                \addplot[
                    color=red,
                    no markers,
                    opacity=0.2,
                    thick
                ]
                table[
                    col sep=comma,
                    header=true,
                    x=epochs,
                    y=fold\n
                ]{data/hypersearch/regime-1-examples/5_train_auc.csv};
            }
            % New Plot Group Begins Here
            \nextgroupplot
            \addplot[
                color=blue,
                no markers,
                ultra thick
            ]
            table[
                col sep=comma,
                header=true,
                x=epochs,
                y=avg
            ]{data/hypersearch/regime-1-examples/6_valid_auc.csv};

            \addplot[
                color=red,
                no markers,
                ultra thick
            ]
            table[
                col sep=comma,
                header=true,
                x=epochs,
                y=avg
            ]{data/hypersearch/regime-1-examples/6_train_auc.csv};

            \foreach \n in {1,...,6} {
                \addplot[
                    color=blue,
                    no markers,
                    opacity=0.2,
                    thick
                ]
                table[
                    col sep=comma,
                    header=true,
                    x=epochs,
                    y=fold\n
                ]{data/hypersearch/regime-1-examples/6_valid_auc.csv};
            }
            \foreach \n in {1,...,6} {
                \addplot[
                    color=red,
                    no markers,
                    opacity=0.2,
                    thick
                ]
                table[
                    col sep=comma,
                    header=true,
                    x=epochs,
                    y=fold\n
                ]{data/hypersearch/regime-1-examples/6_train_auc.csv};
            }
            % New Plot Group Begins Here
            \nextgroupplot
            \addplot[
                color=blue,
                no markers,
                ultra thick
            ]
            table[
                col sep=comma,
                header=true,
                x=epochs,
                y=avg
            ]{data/hypersearch/regime-1-examples/7_valid_auc.csv};

            \addplot[
                color=red,
                no markers,
                ultra thick
            ]
            table[
                col sep=comma,
                header=true,
                x=epochs,
                y=avg
            ]{data/hypersearch/regime-1-examples/7_train_auc.csv};

            \foreach \n in {1,...,6} {
                \addplot[
                    color=blue,
                    no markers,
                    opacity=0.2,
                    thick
                ]
                table[
                    col sep=comma,
                    header=true,
                    x=epochs,
                    y=fold\n
                ]{data/hypersearch/regime-1-examples/7_valid_auc.csv};
            }
            \foreach \n in {1,...,6} {
                \addplot[
                    color=red,
                    no markers,
                    opacity=0.2,
                    thick
                ]
                table[
                    col sep=comma,
                    header=true,
                    x=epochs,
                    y=fold\n
                ]{data/hypersearch/regime-1-examples/7_train_auc.csv};
            }
            % New Plot Group Begins Here
            \nextgroupplot
            \addplot[
                color=blue,
                no markers,
                ultra thick
            ]
            table[
                col sep=comma,
                header=true,
                x=epochs,
                y=avg
            ]{data/hypersearch/regime-1-examples/8_valid_auc.csv};

            \addplot[
                color=red,
                no markers,
                ultra thick
            ]
            table[
                col sep=comma,
                header=true,
                x=epochs,
                y=avg
            ]{data/hypersearch/regime-1-examples/8_train_auc.csv};

            \foreach \n in {1,...,6} {
                \addplot[
                    color=blue,
                    no markers,
                    opacity=0.2,
                    thick
                ]
                table[
                    col sep=comma,
                    header=true,
                    x=epochs,
                    y=fold\n
                ]{data/hypersearch/regime-1-examples/8_valid_auc.csv};
            }
            \foreach \n in {1,...,6} {
                \addplot[
                    color=red,
                    no markers,
                    opacity=0.2,
                    thick
                ]
                table[
                    col sep=comma,
                    header=true,
                    x=epochs,
                    y=fold\n
                ]{data/hypersearch/regime-1-examples/8_train_auc.csv};
            }
            % New Plot Group Begins Here
            \nextgroupplot
            \addplot[
                color=blue,
                no markers,
                ultra thick
            ]
            table[
                col sep=comma,
                header=true,
                x=epochs,
                y=avg
            ]{data/hypersearch/regime-1-examples/9_valid_auc.csv};

            \addplot[
                color=red,
                no markers,
                ultra thick
            ]
            table[
                col sep=comma,
                header=true,
                x=epochs,
                y=avg
            ]{data/hypersearch/regime-1-examples/9_train_auc.csv};

            \foreach \n in {1,...,6} {
                \addplot[
                    color=blue,
                    no markers,
                    opacity=0.2,
                    thick
                ]
                table[
                    col sep=comma,
                    header=true,
                    x=epochs,
                    y=fold\n
                ]{data/hypersearch/regime-1-examples/9_valid_auc.csv};
            }
            \foreach \n in {1,...,6} {
                \addplot[
                    color=red,
                    no markers,
                    opacity=0.2,
                    thick
                ]
                table[
                    col sep=comma,
                    header=true,
                    x=epochs,
                    y=fold\n
                ]{data/hypersearch/regime-1-examples/9_train_auc.csv};
            }
        \end{groupplot}
    \end{tikzpicture}
    \caption{Random examples of models from hyperparameter search regime I.}
    \label{graph:regime_I_examples}
\end{figure}

% Template for a TiKZ/PGFPlot Graph
\begin{figure}[H]
    \begin{tikzpicture}[trim axis left]
        % All the graphing elements are inside axis environment
        \begin{axis}[
            width=\textwidth,
            height=\textwidth,
            scale only axis,
            title={Hyperparameter Search Regime I ($k = 6 $)},
            xlabel={Epochs},
            ylabel={AUROC},
            xmin=1, xmax=20,
            ymin=0.5, ymax=1,
            grid=both,
            minor tick num=1,
            grid style=dotted,
            legend pos=north east,
            x tick label style={
              /pgf/number format/fixed,
              /pgf/number format/fixed zerofill,
              /pgf/number format/precision=0
            },
            y tick label style={
              /pgf/number format/fixed,
              /pgf/number format/fixed zerofill,
              /pgf/number format/precision=2
            },
        ]
            % First graph the validation AUROCs
            \addplot[
                color=blue,
                no markers,
                ultra thick
            ]
            table[
                col sep=comma,
                header=true,
                x=epochs,
                y=avg
            ]{data/hypersearch/regime-1-examples/1_valid_auc.csv};
            \addlegendentry{Avg. Validation AUROC}

            \addplot[
                color=red,
                no markers,
                ultra thick
            ]
            table[
                col sep=comma,
                header=true,
                x=epochs,
                y=avg
            ]{data/hypersearch/regime-1-examples/1_train_auc.csv};
            \addlegendentry{Avg. Training AUROC}

            % Highest average validation AUROC
            \addplot[
              color=blue,
              no marks,
              dotted,
              ultra thick,
              domain=0:50
            ]
            {
                0.808
            };
            \addlegendentry{$y = 0.808$}

            \foreach \n in {1,...,6} {
                \addplot[
                    color=blue,
                    no markers,
                    opacity=0.2,
                    thick
                ]
                table[
                    col sep=comma,
                    header=true,
                    x=epochs,
                    y=fold\n
                ]{data/hypersearch/regime-1-examples/1_valid_auc.csv};
            }   

            \foreach \n in {1,...,6} {
                \addplot[
                    color=red,
                    no markers,
                    opacity=0.2,
                    thick
                ]
                table[
                    col sep=comma,
                    header=true,
                    x=epochs,
                    y=fold\n
                ]{data/hypersearch/regime-1-examples/1_train_auc.csv};
            }

        \end{axis}
    \end{tikzpicture}
    \caption{Best performing model in Regime I}
    \label{graph:regime_I_best_model}
  \end{figure}
  

\subsection{Hyperparameter Search Regime II}

\begin{listing}[H]
    \begin{minted}[
        baselinestretch=1.0,
        frame=lines,
        mathescape,
        autogobble,
        fontsize=\footnotesize,
        style=default,
        breaklines,
        breakbytoken
    ]{python}
    def learning_rate_gridsearch(kfolds: int = 6) -> list[dict[str, Union[int, float, list[tf.keras.callbacks.History]]]]:
        # Grid i: $1.0 \times 10^{-1} \leq$ learning_rate $\leq 1.0 \times 10^{-4}$
        learning_rates: list = [1 * np.float_power(10, -exp) for exp in range(1, 5)]
        # Grid j: $1.0 \times 10^{-1} \leq$ epsilon_rate $\leq 1.0 \times 10^{-8}$
        epsilon_rates : list = [1 * np.float_power(10, -exp) for exp in range(1, 9)]

        search_results: list[dict[str, Union[int, float, list[tf.keras.callbacks.History]]]] = []
        for i, learning_rate in enumerate(learning_rates):
            for j, epsilon_rate in enumerate(epsilon_rates):
                # Conduct K-Fold Experiment
                k_fold_results: list[tf.keras.callbacks.History] = cross_validate(
                    TransferLearningModel,
                    ds_train_and_valid,
                    k=kfolds,
                    epochs=EPOCHS,
                    batch_size=BATCH_SIZE,
                    model_kwargs={"dropout_rate": DROPOUT_RATE}
                    optimizer_kwargs={"learning_rate": learning_rate, "epsilon": epsilon_rate},
                )
                search_results.append({
                    "learning_rate": learning_rate,
                    "epsilon_rate" : epsilon_rate,
                    "history_list" : k_fold_results
                })

        return search_results
    \end{minted}
\caption{Hyperparameter Search Regime II (\href{https://github.com/ShenZhouHong/radiography-ai-project/blob/master/python/hyperparam-search/regime-2.ipynb}{Github})}\label{listing:regime-2}
\end{listing}

% Logorithmic mesh plot for Regime II
% See https://tex.stackexchange.com/questions/552529/plotting-a-matrix-plot-with-an-axis-in-logarithmic-scale-with-pgfplots
% OR https://tikz.dev/pgfplots/reference-3dplots#pgfp./pgfplots/matrix:plot
% Template for a TiKZ/PGFPlot Graph
\begin{figure}[H]
    \begin{tikzpicture}[trim axis left]
        % All the graphing elements are inside axis environment
        \begin{axis}[
            width=\textwidth,
            height=7cm,
            scale only axis,
            title={Regime II Hyperparameter Grid Search Results},
            xlabel={Learning Rate},
            ylabel={Epsilon ($\epsilon$)},
            xmode=log,
            ymode=log,
            grid=both,
            minor tick num=1,
            grid style=dotted,
            enlarge x limits=false,
            enlarge y limits=false,
            mesh/ordering=y varies,
            colormap name=viridis,
            point meta min=0.5,
            point meta max=0.8,
            colorbar,
            colormap/viridis,
            colorbar horizontal=true,
            colorbar style={
                xlabel={Max Valid. AUC},
                x tick label style={
                    /pgf/number format/fixed,
                    /pgf/number format/fixed zerofill,
                    /pgf/number format/precision=2,
                },
            },
            nodes near coords={\pgfmathprintnumber\pgfplotspointmeta},
            every node near coord/.append style={
                xshift=0pt,
                yshift=-7pt,
                black,
                font=\footnotesize,
                /pgf/number format/fixed,
                /pgf/number format/fixed zerofill,
                /pgf/number format/precision=4,
            },
        ]
        \addplot [
            matrix plot*,
            mesh/cols=4,
            mesh/rows=8,
            point meta=explicit
        ] table[
            col sep=comma,
            header=true,
            x=learning_rate,
            y=epsilon_rate,
            meta=max_val_auc
          ]{data/hypersearch/regime-2.csv};
        \end{axis}
    \end{tikzpicture}
    \caption{Results for the Hyperparameter Search Regime II}
    \label{graph:regime_II}
\end{figure}
  

\subsection{Final Hyperparameters}

\section{Final Model Performance}
