\chapter{Conclusion}

In conclusion, this study explores the process and validity of building an AI model to automatically characterise fracture union, using the \emph{radiographic union score for tibial fractures}. By taking pairs of radiographs as inputs, and inferring a RUST score as output, we hope to advance the intersection of medical imaging and machine learning. Although a great deal of literature exists in the use of AI and machine learning models for the automatic detection of fractures, and other anomaly-detection tasks, comparatively little work has been done to date in the characterisation of fractures --- especially via an existing clinical instrument like RUST. Likewise, due to the comparatively smaller nature of datasets in the field of medical imaging, we faced challenges specific to small-datasets, which we hoped to overcome with the method of transfer learning. 

We began by validating transfer-learning as a method, by training and evaluating our base model directly on the RUST dataset. Next, we evaluated two different set of weights (i.e. pre-training sources) for our application, being RadImageNet and ImageNet. After selecting the better-performing ImageNet weights, we conducted a hyperparameter search regime, ultimately finding the best dropout, batch size, learning rate, and epsilon. With the best-performing hyperparameters in hand, we proceeded to train a final model, which we then evaluated on our hold-out test set, yielding an AUROC of \(0.890\).

Using the data that we gathered in our experiments, we were able to examine them critically and contextualise our results in the basis of prior work in the literature. We acknowledge that although an AUROC of \(0.890\) approaches the same percentile as previous works, our model suffers from a low recall, despite the high precision and AUROC. However, the fact that we were able to achieve such a level of performance even in light of a small dataset, indicates that this direction of inquiry is not without merit, and indeed better model design, the use of alternative performance metrics, and further data --- could offer more significant performance gains.

Ultimately as radiography plays an increasing role not just in fixation, but also in rehabilitation, the use of AI can become a powerful tool in alleviating workloads for clinicians, and reducing healthcare cost for patients.
By developing AI models which can accomplish routine tasks like scoring radiographs, we hope to empower clinicians in their practice, ultimately helping \emph{to fulfil the promise of medicine.}