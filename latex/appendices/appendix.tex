% Update page geometry for the appendices, so that we have more room
\newgeometry{
    textwidth=0.8\paperwidth,
    textheight=0.8\paperheight,
}
% Make sure that the header and footer overhang into the marginnotes area
% We must call this every time we update the page geometry, otherwise the
% values WILL be stale!
\setlength{\headwidth}{\textwidth}

\chapter{Additional Materials}

\section{Project Code and Github Repository}

All of the implementation details, model architecture, and data are made available in this project's Git repository (\href{https://github.com/ShenZhouHong/radiography-ai-project/}{Github}). Every code listing contains a link to the specific implementation, and different experiments also contain links to their corresponding Jupyter notebooks where the code was originally run. As a part of this project's commitment to reproducibility, all Jupyter notebooks are documented, and readers are encouraged to follow along and run the experiments for themselves. For further information, please see the repository \mintinline{python}{README.md}.

\url{https://github.com/ShenZhouHong/radiography-ai-project/}

\subsection{Initial Evaluation Models}

Jupyter notebooks used to run the initial evaluations of LeNet 1998, InceptionV3 with end-to-end training, and initial transfer learning models:

\url{https://github.com/ShenZhouHong/radiography-ai-project/tree/master/python/initial-evaluation}

\subsection{Hyperparameter Search Code}

Jupyter notebooks used to perform the hyperparameter search regime.

\url{https://github.com/ShenZhouHong/radiography-ai-project/tree/master/python/hyperparam-search}

\subsection{Analysis Notebooks}

Jupyter notebooks used to analyse the raw data, process for insights and visualisations, and output CSV files:

\url{https://github.com/ShenZhouHong/radiography-ai-project/tree/master/python/analysis}