% Configuration for 'fancy' headers and footers
\usepackage{fancyhdr}               % For fancy headings
\usepackage{lastpage}               % Gives us \lastpage
% Settings used by \usepackage{fancyhdr}
\fancypagestyle{plain}{
  % Clear all definitions from the fancy pagestyle
  \fancyhf{}
  \renewcommand{\headrulewidth}{0pt}
  \renewcommand{\footrulewidth}{0pt}

  % Ensure page numbers are the form page n of m.
  \fancyfoot[C]{Page~\thepage~of~\pageref*{LastPage}}
}
\fancypagestyle{fancy}{
  % Fancy Header Formatting
  \renewcommand{\headrulewidth}{0.4pt}
  \fancyhead[L]{Final Project}
  \fancyhead[C]{}
  \fancyhead[R]{Goldsmiths, University of London}

  % Fancy Footer Formatting
  \renewcommand{\footrulewidth}{0pt}
  \fancyfoot[L]{}
  \fancyfoot[C]{Page~\thepage~of~\pageref*{LastPage}}
  \fancyfoot[R]{}
}
% Update the plain heading format so that the first page includes page n of m
% Set the document header-footer pagestyle to fancy, from above
\pagestyle{fancy}

% Mathematical typesetting packages
% \usepackage{amsmath}                % Needed for most math things.
% \usepackage{amssymb}                % Additional mathematical symbols
% \usepackage{amsthm}                 % For theorem and proof environments
% \usepackage{tkz-euclide}            % Used for planar geometry (Euclidean)

% Scientific graphics and plotting
\usepackage{tikz}                   % Used for graphical illustrations.
\usepackage{pgfplots}               % Used for scientific graphs and charts
\usepgfplotslibrary{groupplots}

% Packages for typesetting code and pseudocode
% In order to use minted, you must edit your makefile to -use-shell-escape!
\usepackage{minted}                 % Code highlighting: \begin{minted}{python}
% \usepackage{algorithm}              % Float environment for pseudocode
% \usepackage{algpseudocode}          % Typesetting library for pseudocode

% Optional LaTeX packages for additional functionality
% \usepackage[noframe]{showframe}     % Debug option to show margin frames.
\usepackage{float}                  % For arranging floats
\usepackage{graphicx}               % Required for embedding images
\usepackage[export]{adjustbox}
\usepackage{booktabs}               % For prettier tables
\usepackage{tabularx}               % Auto scale tables to \textwidth
\usepackage{datetime2}
% \usepackage{geometry}               % Sets more "reasonable" margin-sizes
% \usepackage{xeCJK}                  % For typesetting CJK characters
% \usepackage[l2tabu, orthodox]{nag}  % Verbose warnings for typesetting

% We import the inconsolata font and set it as our main mono font, so that our
% minted code listings will look better.
\usepackage{inconsolata}
\setmonofont[
  AutoFakeSlant,
  BoldItalicFeatures={FakeSlant},
  UprightFont = *-Regular,
  BoldFont = *-Bold
]{Inconsolatazi4}

% % Custom geometry for larger margin notes with asymmetric body layout
% \newgeometry{
%   % Preserve the LaTeX default \textwidth and \textheight
%   textwidth =\textwidth,
%   textheight=\textheight,
%   % Include the margin notes space when doing body calculations
%   includemp=true,
%   % Increase margin notes width and seperation
%   marginparwidth=4cm,
%   marginparsep=0.5cm,
%   % Center doucment body vertically and horizontally
%   hcentering=true,
%   vcentering=true,
%   % Minor tweaks to header and footer seperation
%   headsep   =0.5cm,
%   footskip  =1cm,
% }
%
% % Make sure that the header and footer overhang into the marginnotes area
% % We must call this every time we update the page geometry, otherwise the
% % values WILL be stale!
% \setlength{\headwidth}{\textwidth}
% \addtolength{\headwidth}{\marginparsep}
% \addtolength{\headwidth}{\marginparwidth}

% The biblatex package should go last!


\usepackage[style=ieee, backend=biber]{biblatex} % Nice MLA bibliography
\addbibresource{bibliography.bib} % Biblatex. See includes/formatting.tex
\SetCiteCommand{\autocite} % For use with csquotes, where you can do:
% \textcquote[175]{key}{quoted text here}

% Custom IEEE biblatex environment with protruded label numbers
\defbibenvironment{ieee-protrusion}
  {\list{}
     {\printtext[labelnumberwidth]{%
        \printfield{labelprefix}%
        \printfield{labelnumber}}}
     {\setlength{\labelwidth}{\labelnumberwidth}%
      \setlength{\leftmargin}{-\labelsep}%
      \setlength{\labelsep}{\biblabelsep}%
      \addtolength{\leftmargin}{0.5em}%
      \setlength{\itemsep}{\bibitemsep}%
      \setlength{\parsep}{\bibparsep}}%
      \renewcommand*{\makelabel}[1]{\hss##1}}
  {\endlist}
  {\item}

% Any additional user configuration goes below
