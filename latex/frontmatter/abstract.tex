\chapter{Abstract}

The use of AI models in the processing and interpretation of medical imagery is an emerging field, particularly in the domain of anomaly detection and localisation. However comparatively little work has been done with the characterisation of anomalies, such as the scoring of radiographs with standard clinical instruments. This study explores the use of transfer-learning as a method to develop an AI model that is able to score radiographs of long-bone fractures using Whelan et al's \emph{Radiographic Union Score for Tibial Fractures}. By first training a baseline model using end-to-end training, the validity of transfer-learning as an approach is validated. Next, two different models with the same architecture, but different weights, were evaluated and compared. Finally, taking the best-performing ImageNet weights, a hyperparameter search was conducted, yielding optimal hyperparameters for a final model. This model achieved an AUROC of \(0.890\) on the hold-out evaluation set, demonstrating persuasive performance despite the limited data available. 

\textsc{Keywords:} transfer-learning, medical imaging, radiography, union, machine learning, AI, computer vision.